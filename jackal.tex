\documentclass[]{clearpath-manual}

\begin{document}

\manualcover{gen/cover-page.pdf}
\tableofcontents

\section{Introduction}

Jackal is a rugged, light, fast, and easy-to-use unmanned ground vehicle for ROS
Indigo, presented by Clearpath Robotics.

Jackal includes a standard internal PC, as well as basic IMU and GPS. Standard
perception modules are available, including URDF and simulator integration, and
demonstration applications.

Please inquire with Clearpath Robotics for details.

\begin{note}[Note Title]
Lorem ipsum dolor sit amet, consectetur adipiscing elit. Vestibulum id
nibh iaculis, scelerisque quam nec, congue est. Praesent sollicitudin viverra odio
eu tincidunt. Ut semper volutpat metus, ut feugiat metus lobortis eu. Fusce tincidunt
rutrum posuere. Nunc consequat luctus bibendum.
\end{note}

\begin{note}
Note without a title. Lorem ipsum dolor sit amet, consectetur adipiscing elit. Vestibulum id
nibh iaculis, scelerisque quam nec, congue est. Praesent sollicitudin viverra odio
eu tincidunt. Ut semper volutpat metus, ut feugiat metus lobortis eu. Fusce tincidunt
rutrum posuere. Nunc consequat luctus bibendum.
\end{note}

\subsection{What's Included}

Contained in your Jackal shipment are the following items:

\begin{itemize}
  \item Jackal UGV
  \item 270 watt-hour lithium battery pack
  \item 110V/220V universal charger
  \item Sony Bluetooth controller
  \item Jackal User Manual
\end{itemize}

If you elected to purchase standard upgrade modules, these will be integrated into your Jackal chassis.

\subsection{Hardware Overview}

Jackal's external features include the 100mm square mounting pattern on the top plate, 95mm wheels, human
machine interface panel (HMI), and top plate latches. These features are labeled in \autoref{ext}.

\begin{figure}[ht]
  \centering
  \placeholder{9cm}{5cm}
  \caption{External components of Jackal.}
  \label{ext}
\end{figure}

To access Jackal's interior, actuate the latches under the front end of the lid, on the opposite end from
the HMI. When you lift the lid, you will see Jackal's onboard Li-Ion battery pack, and its two connectors.
The large Anderson Power Pole connector is to supply power to Jackal and must be connected in order for
Jackal to operate. The smaller white Molex connector allows the battery pack to be charged inside Jackal 
while Jackal is powered off. It is recommended to connect both. The interior components of Jackal are
labeled in \autoref{int}.

\begin{figure}[ht]
  \centering
  \placeholder{9cm}{5cm}
  \caption{Battery area inside Jackal.}
  \label{int}
\end{figure}

Finally, you may undo the thumbscrews which hold Jackal's computer tray to the lid. The tray lowers,
revealing Jackal's onboard Mini-ITX PC, user power supplies, and internal user hardware mounting area.
Please see \autoref{tray} and \autoref{upb} for the components of the tray and user power supplies.
Note that the fused user power is available as four-pin Molex connectors, or a plug-in screw terminal
block. For more information on integrating payloads electrically, see \autoref{payload-elec}.

\begin{figure}[pt]
  \centering
  \placeholder{12cm}{8cm}
  \caption{Computer and user tray.}
  \label{tray}
\end{figure}

\begin{figure}[pb]
  \centering
  \placeholder{12cm}{8cm}
  \caption{User power supply.}
  \label{upb}
\end{figure}

\subsection{System Architecture}

Like many ROS robots, Jackal is built around an x86 PC running Ubuntu, paired with an
ARM MCU. The MCU handles IO, power supply monitoring, and motor control, as well as
supplying data from the integrated IMU and GPS receiver. The communication channel
between the MCU and PC is a Full Speed USB connection, with the MCU operating as a
standard serial CDC device.

The communication protocol used is rosserial. An instance of the rosserial\_server
serial node is embedded in the jackal\_base node, where it is connected directly to
Jackal's kinematic controller. \autoref{arch} is a high-level overview of the ROS
nodes in base Jackal and the data flow through them.

\begin{figure}[hb]
  \centering
  \placeholder{16cm}{18cm}
  \caption{ROS nodes and topics on Jackal.}
  \label{arch}
\end{figure}

\section{Getting Started}

The first step is to power up your Jackal and have some fun driving it around! If you've
just unpacked Jackal from its shipment packaging, you'll need to install and plug in the
battery.

Begin by pressing the power button on Jackal's HMI panel. The LEDs should show a test
pattern, after which you can wait about 30 seconds for the internal PC to finish booting
up.

When the comms LED ( \placeholder{0.35cm}{0.35cm} ) is green, this signals that the PC is
finished booting up, and that the PC and MCU are in communication. At this point, press
the PS button on the Sony Bluetooth controller, to sync the controller to Jackal. Now,
you should be able to drive around. Hold the L2 trigger button, and push the joystick
forward.

If you're not seeing any action, check \nameref{trouble} on page \pageref{trouble}.

\subsection{Talking to Jackal}

The next thing you probably want to do is get Jackal on your wireless network. Once it's
on the wireless, you can access it via SSH or as a remote ROS master.


\subsection{Jackal Desktop}

To command or observe Jackal from your desktop computer, first set up a basic
ROS Installation. See the following page for details:

\url{http://wiki.ros.org/indigo/Installation/Ubuntu}

When your ROS install is set up, install the Jackal desktop packages:

\begin{lstlisting}
sudo apt-get install ros-indigo-jackal-desktop
\end{lstlisting}

To connect to a running Jackal, follow these steps:


\section{Apps}

Jackal's demonstration software are packaged as Robot Apps, or \textit{rapps}.


\section{Charging \& Maintenance}

Jackal's Li-Ion battery pack may be charged internal to the chassis---simply plug in
the charger to the charge port located under the rear fender. Charging will occur
only when Jackal is powered down.

Alternatively, if you have multiple battery packs, you can easily lift the lid and
remove the battery for external charging. When charging externally, remove the pigtail
which adapts the charger to the platform's weather sealed charge port.

TODO: More about charging and battery care.


\section{Payload Integration Guide}

If you're wanting to attach custom hardware to Jackal, you'll have to take care of
mechanical mounting, electrical supply, and software integration. This section
aims to equip you with respect to these challenges.

\subsection{Mechanical Mounting}

For external payloads, use the 

TODO: Orthographic diagram of that.

\subsection{User Electrical Supply}\label{payload-elec}

TODO: Diagram of pinout of user power Molex connection and screw terminal.

\subsection{Software Integration}

TODO: Links to ROS wiki, information about URDF, etc.


\section{Troubleshooting}\label{trouble}

Clearpath is committed to your success with Jackal. Please get in touch with us at
\href{mailto:support@clearpathrobotics.com}{support@clearpathrobotics.com} and we'll do our
best to get you rolling again quickly.

If your issue is specifically about ROS and is something which may be of interest
to the broader community, consider asking it on \href{http://answers.ros.org}{answers.ros.org}.
If you don't get a satisfactory response, please ping us with a link to your question.

TODO: Suggestions for mechanical/electrical troubleshooting. Wheel replacement? MCU replacement? 


\end{document}
